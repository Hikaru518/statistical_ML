\documentclass[]{book}

%These tell TeX which packages to use.
\usepackage{array,epsfig}
\usepackage{amsmath}
\usepackage{amsfonts}
\usepackage{amssymb}
\usepackage{amsxtra}
\usepackage{amsthm}
\usepackage{mathrsfs}
\usepackage{color}
\usepackage{graphicx}
\usepackage{float}

%Algorithm packages
\usepackage{algorithm}  
\usepackage{algpseudocode}  
\usepackage{amsmath}  
\renewcommand{\algorithmicrequire}{\textbf{Input:}}  % Use Input in the format of Algorithm  
\renewcommand{\algorithmicensure}{\textbf{Output:}} % Use Output in the format of Algorithm  
%Here I define some theorem styles and shortcut commands for symbols I use often
\theoremstyle{definition}
\newtheorem{defn}{Definition}
\newtheorem{thm}{Theorem}
\newtheorem{cor}{Corollary}
\newtheorem*{rmk}{Remark}
\newtheorem{lem}{Lemma}
\newtheorem*{joke}{Joke}
\newtheorem{ex}{Example}
\newtheorem*{soln}{Solution}
\newtheorem{prop}{Proposition}

\newcommand{\lra}{\longrightarrow}
\newcommand{\ra}{\rightarrow}
\newcommand{\surj}{\twoheadrightarrow}
\newcommand{\graph}{\mathrm{graph}}
\newcommand{\bb}[1]{\mathbb{#1}}
\newcommand{\Z}{\bb{Z}}
\newcommand{\Q}{\bb{Q}}
\newcommand{\R}{\bb{R}}
\newcommand{\C}{\bb{C}}
\newcommand{\N}{\bb{N}}
\newcommand{\M}{\mathbf{M}}
\newcommand{\m}{\mathbf{m}}
\newcommand{\MM}{\mathscr{M}}
\newcommand{\HH}{\mathscr{H}}
\newcommand{\Om}{\Omega}
\newcommand{\Ho}{\in\HH(\Om)}
\newcommand{\bd}{\partial}
\newcommand{\del}{\partial}
\newcommand{\bardel}{\overline\partial}
\newcommand{\textdf}[1]{\textbf{\textsf{#1}}\index{#1}}
\newcommand{\img}{\mathrm{img}}
\newcommand{\ip}[2]{\left\langle{#1},{#2}\right\rangle}
\newcommand{\inter}[1]{\mathrm{int}{#1}}
\newcommand{\exter}[1]{\mathrm{ext}{#1}}
\newcommand{\cl}[1]{\mathrm{cl}{#1}}
\newcommand{\ds}{\displaystyle}
\newcommand{\vol}{\mathrm{vol}}
\newcommand{\cnt}{\mathrm{ct}}
\newcommand{\osc}{\mathrm{osc}}
\newcommand{\LL}{\mathbf{L}}
\newcommand{\UU}{\mathbf{U}}
\newcommand{\support}{\mathrm{support}}
\newcommand{\AND}{\;\wedge\;}
\newcommand{\OR}{\;\vee\;}
\newcommand{\Oset}{\varnothing}
\newcommand{\st}{\ni}
\newcommand{\wh}{\widehat}

%Pagination stuff.
\setlength{\topmargin}{-.3 in}
\setlength{\oddsidemargin}{0in}
\setlength{\evensidemargin}{0in}
\setlength{\textheight}{9.in}
\setlength{\textwidth}{6.5in}
\pagestyle{empty}



\begin{document}


\begin{center}
{\Large COMP 540 \hspace{0.5cm} HW 5}\\
\textbf{Peiguang Wang, Xinran Zhou}\\ %You should put your name here
Due: 3/5/2018 %You should write the date here.
\end{center}

\vspace{0.2 cm}


\subsection*{1: Deep neural networks }
\begin{enumerate}
	\item Why do deep neural networks typically outperform shallow networks?
	\begin{soln}
		By using deep neural network and adding more layers, we can approximate function using less parameters.
		The deep network encodes a set of prior beliefs about the structure of the function we want to learn. Thus, the deep nerual networks reduce the amount of data we should use to get a satisfying result.
	\end{soln}
	\item What is leaky RELU activation and why is it used?
	\begin{soln}
		Leaky relu is basically based on relu activation function and tries to fix the 'dying' problem of relu. When $x<0$, the leaky relu has a small slope instead of being zero.\\
		The reason why we use leaky relu is that it can give a small constant gradient when the input falls in the region $x<0$. So it can fix the problem of "dead relu". 
	\end{soln}
	\item In one or more sentences, and using sketches as appropriate, contrast: AlexNet, VGGNet, GoogleNet and ResNet. What is the one defining characteristic of each network?
	\begin{soln}
		AlexNet: AlexNet uses RELU activate function instead of sigmoid function for the first time. And it also introduce a new dropout layer in the network.
		
		VGGNet: VGGNet consists of either 16 or 19 convolutional layers and has very uniform architecture.
		
		GoogleNet: This module is based on several very small convolutions in order to drastically reduce the number of parameters. 
		
		ResNet: ResNet introduces a so called "shortcut connection" that skips one or more layers, which allow the gradients can be backprop to the first layers. This allows us to train a much deeper network up to 152 layers.
	\end{soln}
	
\end{enumerate}
\subsection*{2: Decision trees, entropy and information gain }
	\begin{enumerate}
		\item  Show that $H(S) ≤ 1$ and that $H(S) = 1$ when $p = n$.
		\begin{soln}
			Since
			$$H(q) = - q log(q)- (1-q)log(1-q)$$
			the second derivative of the $-H(q)$ is non-negative, so the negative entropy is convex. The $H(q)$ is concave. The maximum can be obtained at $\frac{\partial H }{\partial q } = 0$
			$$\frac{\partial H }{\partial q } = -log(q) +log(1-q)$$
			thus we got $q = 0.5$, which means that $p = n$ and $H(S) = 1$.
			
			Therefore, $H(S) ≤ 1$ and that $H(S) = 1$ when $p = n$.
			
		\end{soln}
		\item Calculate the reduction in cost using misclassification rate, entropy, and Gini index for models A and B. Which is the preferred split (model A or model B) according to these cost calculations?
		\begin{soln}
			\textbf{Misclassification rate:}
			$$error_A = \frac{100 + 100 }{400+400} = 0.25$$
			$$error_B = \frac{200}{400+400} = 0.25$$ 
			
			\textbf{Entropy:} 
			For both A and B:
			$$H(D) = 1$$
			For A:
			$$H(D_1) =H(D_2) =  - 0.75 log(0.75) - 0.25log(0.25) = 0.811$$
			$$g(D,A) = H(D) -0.5H(D_1) - 0.5H(D_2) = 0.189$$
			For B:
			$$H(D_1) = - \frac{1}{3}log(\frac{1}{3}) - \frac{2}{3}log(\frac{2}{3}) = 0.913$$
			$$H(D_2)= 0$$
			$$g(D,B) = H(D)-0.75H(D_1)-0.25H(D_2) = 0.312$$
			
			\textbf{Gini Index:}
			$$Gini(A) = 0.5(1-0.75^2 - 0.25^2) + 0.25(1-0.25^2 - 0.75^2) = 0.375$$
			$$Gini(B) = 0.75(1-\frac{2}{3}^2 - \frac{1}{3}^2)+0.25(1-1-0)=\frac{1}{3}$$
			Among these three cost calculations, the entropy is the preferred split since the difference between A and B in this cost calculation is the biggest.
		\end{soln}
		\item Can the misclassification rate ever increase when splitting on a feature? If so, give an example. If not, give a proof.
		\begin{soln}
			No, the misclassification rate will not increase when splitting on a feature. 
		\end{soln}
	\end{enumerate}
\subsection*{3: Bagging }
	\begin{enumerate}
		\item Assuming that the individual errors $\epsilon_l$(x) have zero mean and are uncorrelated, that is $E_x[\epsilon_l (x)] = 0$ and $E_x[\epsilon_m (x)\epsilon_l (x)] = 0$ for $m \neq l $, show that
		$$ E_{bag} = \frac{1}{L} E_{av}$$
		\begin{soln}
			Since 
			$$\epsilon_{bag} = \frac{1}{L}\sum_{l=1}^{L}(f(x)+\epsilon_l(x)) - f(x)$$
			where $\epsilon_l ~ N(\mu,\sigma_l ^2)$, and they are uncorrelated 
			If we calculate the $ E_{bag}$, then
			$$E_{bag} = E[\epsilon_{bag}(x)^2] = var(\epsilon_{bag}(x))$$
			the result is $\frac{1}{L^2}\sum_{l=1}^{L} \sigma_L ^2$
			And we have
			$$E_{av} = \frac{1}{L}\sum_{l=1}^{L}E_x[\epsilon_l(x)^2]$$
			Therefore $E_{bag} = \frac{1}{L}E_{av}$
		\end{soln}
	\item  Show that the average expected squared-error $E_{av}$ of the individual functions and the expected error of bagging $E_{bag}$ satisfy $E_{bag} \leq E_{av}$
	\begin{soln}
		Using Jensens inequality for the special case of convex function $f(x) = x^2$, and we suppose $\lambda_l = \frac{1}{L}$
		$$\sum_{l =1}^{L}\lambda_l f(\epsilon_l) = \frac{1}{L}\sum_{l = 1}^{L}\epsilon_l^2$$
		and according to Jensen's inequality,
		$$\sum_{l =1}^{L}\lambda_l f(\epsilon_l) \geq f(\sum_{l =1}^{L}\lambda_l \epsilon_l) = (\frac{1}{L}\sum_{l = 1}^{L}\epsilon_l^2)^2$$
		so
		$$\frac{1}{L}\sum_{l = 1}^{L}\epsilon_l^2 \geq \epsilon_{bag}^2$$
		therefore
		$$E[\frac{1}{L}\sum_{l = 1}^{L}\epsilon_l^2] \geq E[\epsilon_{bag}^2]$$
		take the expectation on both sides $E_av \geq E_bag $
	\end{soln}
	\end{enumerate}

\end{document}


