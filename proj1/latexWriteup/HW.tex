\documentclass[]{book}

%These tell TeX which packages to use.
\usepackage{array,epsfig}
\usepackage{amsmath}
\usepackage{amsfonts}
\usepackage{amssymb}
\usepackage{amsxtra}
\usepackage{amsthm}
\usepackage{mathrsfs}
\usepackage{color}

%Here I define some theorem styles and shortcut commands for symbols I use often
\theoremstyle{definition}
\newtheorem{defn}{Definition}
\newtheorem{thm}{Theorem}
\newtheorem{cor}{Corollary}
\newtheorem*{rmk}{Remark}
\newtheorem{lem}{Lemma}
\newtheorem*{joke}{Joke}
\newtheorem{ex}{Example}
\newtheorem*{soln}{Solution}
\newtheorem{prop}{Proposition}

\newcommand{\lra}{\longrightarrow}
\newcommand{\ra}{\rightarrow}
\newcommand{\surj}{\twoheadrightarrow}
\newcommand{\graph}{\mathrm{graph}}
\newcommand{\bb}[1]{\mathbb{#1}}
\newcommand{\Z}{\bb{Z}}
\newcommand{\Q}{\bb{Q}}
\newcommand{\R}{\bb{R}}
\newcommand{\C}{\bb{C}}
\newcommand{\N}{\bb{N}}
\newcommand{\M}{\mathbf{M}}
\newcommand{\m}{\mathbf{m}}
\newcommand{\MM}{\mathscr{M}}
\newcommand{\HH}{\mathscr{H}}
\newcommand{\Om}{\Omega}
\newcommand{\Ho}{\in\HH(\Om)}
\newcommand{\bd}{\partial}
\newcommand{\del}{\partial}
\newcommand{\bardel}{\overline\partial}
\newcommand{\textdf}[1]{\textbf{\textsf{#1}}\index{#1}}
\newcommand{\img}{\mathrm{img}}
\newcommand{\ip}[2]{\left\langle{#1},{#2}\right\rangle}
\newcommand{\inter}[1]{\mathrm{int}{#1}}
\newcommand{\exter}[1]{\mathrm{ext}{#1}}
\newcommand{\cl}[1]{\mathrm{cl}{#1}}
\newcommand{\ds}{\displaystyle}
\newcommand{\vol}{\mathrm{vol}}
\newcommand{\cnt}{\mathrm{ct}}
\newcommand{\osc}{\mathrm{osc}}
\newcommand{\LL}{\mathbf{L}}
\newcommand{\UU}{\mathbf{U}}
\newcommand{\support}{\mathrm{support}}
\newcommand{\AND}{\;\wedge\;}
\newcommand{\OR}{\;\vee\;}
\newcommand{\Oset}{\varnothing}
\newcommand{\st}{\ni}
\newcommand{\wh}{\widehat}

%Pagination stuff.
\setlength{\topmargin}{-.3 in}
\setlength{\oddsidemargin}{0in}
\setlength{\evensidemargin}{0in}
\setlength{\textheight}{9.in}
\setlength{\textwidth}{6.5in}
\pagestyle{empty}



\begin{document}


\begin{center}
{\Large COMP 540 \hspace{0.5cm} HW 1}\\
\textbf{Peiguang Wang, Xinran Zhou}\\ %You should put your name here
Due: 1/18/2018 %You should write the date here.
\end{center}

\vspace{0.2 cm}


\subsection*{Part 0: Background refresher}

%% Problem 1
\begin{enumerate}
\item\label{norms}
Generate different distributions from uniform distribution:
\begin{enumerate}
	\item	Plot the histogram of a categorical distribution with probabilities [0.2,0.4,0.3,0.1].
	\item	Plot the univariate normal distribution with mean of 10 and standard deviation of 1.
	\item	Produce a scatter plot of the samples for a 2-D Gaussian with mean at [1,1] and a covariance matrix [[1,0.5],[0.5,1]]
	\item	Test your mixture sampling code by writing a function that implements an equal weighted mixture of four Gaussians in 2 dimensions, centered at $(\pm 1; \pm 1)$ and having covariance $I$. Estimate the probability that a sample from this distribution lies within the unit circle centered at (0.1, 0.2).
	
\end{enumerate}

%% Problem 2
\item	Prove that the sum of two independent Poisson random variables is also a Poisson random variable.
\begin{proof}
	The characteristic function of a Poisson random variable is
	$$\varPhi _1(u) = e^{\lambda _1 (e^{iu}-1)}$$
	
	Let $X_1$ and $X_2$ denote two independent Poisson random variables. Let $X=X_1+X_2$
	
	Let $\varPhi _1(u)$ and $\varPhi _2(u)$ denote the characteristic functions of $X_1$ and $X_2$:
	$$\varPhi _1(u) = e^{\lambda _1 (e^{iu}-1)}$$
	$$\varPhi _2(u) = e^{\lambda _2 (e^{iu}-1)}$$
	
	Let $\varPhi (u)$ denote the characteristic functions of $X$. Since $X=X_1 + X_2$, we have:
	$$\varPhi (u) = \varPhi _1(u) \varPhi _2(u)=e^{\lambda _1 (e^{iu}-1)}e^{\lambda _2 (e^{iu}-1)}$$
	Simplify the equation above,
	$$\varPhi (u) = e^{(\lambda _1 + \lambda _2) ( \frac{\lambda_1}{\lambda_1 + \lambda_2}e^{iu} + \frac{\lambda_2}{\lambda_1 + \lambda_2}e^{iu}) - 1 }. $$
	That is
	$$\varPhi (u) = e^{(\lambda_1 + \lambda_2)  (e^{iu}-1)}. $$
	Comparing with the characteristic function of Poisson distribution, we can see that X is also a Poisson random variable.
\end{proof}

%% Problem 3
\item	Let $X_0$ and $X_1$ be continuous random variables. Show that if
	$$P(X_0=x_0) = \alpha_0	e^{-\frac{(x_0-\mu_0)^2}{2 \sigma_0^2}}$$
	$$P(X_1=x_1|X_0=x_0) = \alpha	e^{-\frac{(x_1-x_0)^2}{2 \sigma^2}}$$
there exists $\alpha_1$, $\mu_1$ and $\sigma_1$ such that
$$P(X_1=x_1) = \alpha_1	e^{-\frac{(x_1-\mu_1)^2}{2 \sigma_1^2}}$$	
Write down expressions for these quantities in terms of $\alpha_0$, $\alpha$, $\mu_0$, $\sigma_0$ and $\sigma$.

\begin{soln}
	If X,Y are both Gaussian random variable, then
	$$Y|X=x\sim N\left(\mu_Y+\rho \dfrac{\sigma_Y}{\sigma_X}(X-\mu_X),\quad \sigma^2_Y(1-\rho^2)\right)$$ 
	where $\mu_X$, $\mu_Y$ are mean of $X$ and $Y$; $\sigma_X^2$, $\sigma_Y^2$ are variance of $X$ and $Y$; $\rho$ is the correlation coefficient between $X$ and $Y$.
	
	According to the problem, $X_0$, $X_1$ and $X_1 | X_0$ are all Gaussian. So we have the following equations:
	$$ \left\{
	\begin{aligned}
	\mu_1 + \rho \frac{\sigma_1}{\sigma_0}(x_0 - \mu_0) & =  x_0 ,   for \; all \; x_0 \\ 
	\sigma_1^2  (1-\rho^2) & =  \sigma^2 
	\end{aligned} 
	\right.
	$$

	Solve the equation, then $\sigma_1^2 = \sigma^2 + \sigma_0^2$, $\mu_1 = -\mu_0$. 
	And since $\alpha_1 = \frac{1}{\sqrt{2 \pi} \sigma_1}$, we have 
	$$\alpha_1 = \sqrt{ \frac{1}{(1/\alpha)^2 + (1/\alpha_0)^2}  } $$.
	
\end{soln}

%% Problem 4
\item	Find the eigenvalues and eigenvectors of the following 2 $\times$ 2 matrix $A$.
$$A = \left({
\begin{matrix}

13 & 5 \\ 
2 & 4
\end{matrix}
}\right) $$

\begin{soln}
	Let $\lambda$ and \boldmath $x$ \unboldmath denote the eigenvalue and eigenvector of A. According to the definition of eigenvalue,
	$$A \boldsymbol{x} = \lambda \boldsymbol{x} $$
	Solve the equation to get eigenvalues
	$$ |A-\lambda I| = 0 $$
	That is,
	$$\lambda ^2 -14\lambda +42 = 0$$
	A has two eigenvalues: $\lambda_1 = 14$, $\lambda_2 = 3$.
	
	When $\lambda$ = 14,
	$$(A- \lambda I) \boldsymbol{x}  = \left( {\begin{matrix}
	-1 & 5 \\ 
	2 & -10
	\end{matrix} }\right) \boldsymbol{x} = 0 $$
	$$\boldsymbol{x} = \left({\begin{matrix}
		5 & 1
		\end{matrix} }\right) ^ T $$
	When $\lambda$ = 3,
	$$(A- \lambda I) \boldsymbol{x}  = \left( {\begin{matrix}
		10 & 5 \\ 
		2 & 1
		\end{matrix} }\right) \boldsymbol{x} = 0 $$
	$$\boldsymbol{x} = \left({\begin{matrix}
		1 & -2
		\end{matrix} }\right) ^ T $$
	
	In summary, A has two eigenvalues, $\lambda_1 = 14$, $\lambda_2 = 3$. The corresponding eigenvectors are $\boldsymbol{x_1} = \left({\begin{matrix}
		5 & 1
		\end{matrix} }\right) ^ T $ and $\boldsymbol{x_2} = \left({\begin{matrix}
		1 & -2
		\end{matrix} }\right) ^ T $.
	
\end{soln}

%% Problem 5
\item	Provide one example for each of the following cases, where A and B are 2 × 2 matrices.
\begin{enumerate}
	\item  $(A + B)^2 \neq A^2 + 2AB + B^2$
	\item  $AB = 0, A \neq 0, B \neq 0$
\end{enumerate}

\begin{soln}
	\begin{enumerate}
		\item one example that satisfies (a) is:
		$$A=\left({\begin{matrix}
			0 & 0 \\ 
			0 & 1
			\end{matrix} }\right), 
		B = \left({\begin{matrix}
			0 & 1 \\ 
			0 & 0
			\end{matrix} }\right)$$
		Calculate left,
		$$left = (A+B)^2 = \left({\begin{matrix}
			0 & 1 \\ 
			0 & 0
			\end{matrix} }\right) \left({\begin{matrix}
			0 & 1 \\ 
			0 & 0
			\end{matrix} }\right) = \left({\begin{matrix}
			0 & 1 \\ 
			0 & 1
			\end{matrix} }\right)$$
		Calculate right,
		$$right = A^2 + 2AB + B^2 = \left({\begin{matrix}
			0 & 0 \\ 
			0 & 1
			\end{matrix} }\right) + \boldsymbol{0} + \boldsymbol{0} = \left({\begin{matrix}
			0 & 0 \\ 
			0 & 1
			\end{matrix} }\right)$$
		And $left \neq right$
		
		\item 	one example that satisfies (b) is:
		$$A=\left({\begin{matrix}
			0 & 0 \\ 
			0 & 1
			\end{matrix} }\right), 
		B = \left({\begin{matrix}
			0 & 1 \\ 
			0 & 0
			\end{matrix} }\right)$$
		where $A\neq 0$, and $B \neq 0$. Calculate $AB$,
		$$AB = \left({\begin{matrix}
			0 & 0 \\ 
			0 & 1
			\end{matrix} }\right) \left({\begin{matrix}
			0 & 1 \\ 
			0 & 0
			\end{matrix} }\right) = \boldsymbol{0} $$		
	\end{enumerate}
\end{soln}

\item Let $u$ denote a real vector normalized to unit length. That is, $u^T u = 1$. Show that
$$A = I - 2 u u^T $$
is orthogonal, i.e., $A^T A = 1$.
\begin{proof}
	Derive from left,
	$$A^T A = (I-2u u^T)^T(I-2u u^T) =(I-2u u^T)(I-2u u^T) = I-2u u^T - 2u u^T + 4 u u^T = I$$
	So $left = right$.	
\end{proof}

\end{enumerate}

%% part1
\subsection*{Part 1: Locally weighted linear regression}
%% part1 problem 1
\begin{enumerate}
	\item Show that $J(\theta)$ can be written in the form
	$$J(\theta)=(X \theta - y)^T W (X \theta - y)$$
	for an appropriate diagonal matrix $W$, where $X$ is the $m \times d$ input matrix and y is a $m \times 1$ vector denoting the associated outputs. State clearly what $W$ is.
\begin{proof}
	We know that $J(\theta)$ can also be written as 
	$$J(\theta)=\frac{1}{2} \sum_{i=1}^{m} w^{(i)} (\theta^T x^{(i)} - y^{(i)})^2$$
	where $x^{(i)}$ is $d \times 1$ vector and $\theta$ is a $d \times 1$ vector. We consider each row of the matrix $X$ as as a $1 \times d$vector $x^i$, so we can write $X = [x^1, x^2, ..., x^m]^T$. So
	$$J(\theta)=(X \theta - y)^T W (X \theta - y)
	=[x^1 \theta - y^1, x^2 \theta - y^2,...,x^m \theta - y^m] W [x^1 \theta - y^1, x^2 \theta - y^2,...,x^m \theta - y^m]^T$$
	So $W$ is a $m \times m$ diagonal matrix
	$$W=\left(\begin{matrix}
	2 w^{(1)} & 0 & 0 & 0 \\ 
		0 & 2 w^{(2)} & 0 & 0 \\ 
		0 & 0 & ... & 0  \\ 
		0 & 0 & 0 & 2 w^{(m)}
	
	\end{matrix}\right)$$ 
\end{proof}
	
%%part 1 problem 2
	\item If all the $w^{(i)}$'s are equal to 1, the normal equation to solve for the parameter $\theta$ is:
	$$X^T X \theta = X^T y$$
	and the value of $\theta$ that minimizes $J(\theta)$ is $(X^T X)^{-1} X^T y$. By computing the derivative of the weighted $J(\theta)$ and setting it equal to zero, generalized the normal equation to the weighted setting and solve for $\theta$ in closed form in terms of $W$, $X$ and $y$.
\begin{proof}
	$$J(\theta)=(X \theta - y)^T W (X \theta - y)=\theta^T X^T W X \theta^T - \theta^T X^T W y - y^T W X \theta + y^T W y$$
	Compute the derivative of $J(\theta)$
	$$\frac{\partial{J(\theta)}}{\partial(\theta)}=2 X^T W X \theta -X^T W y - X^T W^T y$$
	Since $W$ is a diagonal matirx $W = W^T$, the equation can be written as
	$$\frac{\partial{J(\theta)}}{\partial(\theta)} = 2 X^T W X \theta -2 X^T W y$$
	By setting it equal to zero, we can find the value of $\theta$ that minimizes $J(\theta)$, the equation is :
	$$X^T W X \theta = X^T W y $$
	So the value of $\theta$ in form in terms of $W$, $X$ and $y$ is $(X^T W X)^{-1} X^T W y$.
\end{proof} 

%% part 1 problem 3
	\item To predict the target value for an input vector $x$, one choice for the weighting function $w^{(i)}$ is:
	$$w^{(i)}= \exp({-\frac{(x -x^{(i)})^T (x-x^{(i)})}{2 \tau^2}})$$
	%% big 
	Points near $x$ are weighted more heavily than points far away from $x$. The parameter $\tau$ is a band width defining the sphere of influence around $x$. Note how the weights are defined by the input $x$. Write down an algorithm for calculating $\theta$ by batch gradient descent for locally weighted linear regression. Is locally weighted linear regression a parametric or a non-parametric method?
\end{enumerate}

%%part 2 
\subsection*{Part 2: Properties of the linear regression estimator}

\begin{enumerate}
%% part 2 problem 1
	\item Show that $E[\theta]= \theta^*$ for the least squares estimator.
	
\begin{proof}
	In part 1 problem 2, we can get thee value of $\theta$ given the normal equation $X^T X \theta = X^T y$ is 
	$$\theta = (X^T X)^{-1} X^T y$$
	The date comes from the linear model:
	$$y^{(i)} = \theta^T x^{(i)} + \epsilon^{(i)} $$
	the expectation of $\theta$ is
	 \begin{equation*}
		\begin{split}
		E[\theta] &= E[(X^T X)^{-1} X^T y]\\
				 &= E[(X^T X)^{-1} X^T (X\theta^* + \epsilon)]\\
				 &=E[(X^T X)^{-1} (X^T X \theta^* +X^T \epsilon)]\\
				 &=E[(X^T X)^{-1}X^T X\theta^* + (X^T X)^{-1} X^T \epsilon]\\
				 &=E[\theta^*] +E[(X^T X)^{-1} X^T \epsilon]
		\end{split}
		\end{equation*}
	since each $\epsilon^{(i)}$is an independent random variable drawn from a normal distribution with zero mean and variance $\sigma^2$ and $\theta^*$ is a true parameter that has certain value. Then $E[\theta]$ can be written as
	$$E[\theta] = E[\theta^*] +0 = \theta^*$$
	
\end{proof}

%%part 2 problem 2
	\item Show that the variance of the least squares estimator is $Var(\theta) = (X^T X)^{-1} \sigma^2$.
	
\begin{proof}
	$$Var(\theta) = E[\theta^2] -(E[\theta])^2$$
	since we already knew that $E[\theta] = \theta^*$. So in order to get $Var(\theta)$, all we need to do is to compute $E[\theta^2]$.
	\begin{equation*}
	\begin{split}
	E[\theta^2] &= E[((X^T X)^{-1} X^T y)((X^T X)^{-1} X^T y)^T]\\
	&=E[(\theta^* + (X^T X)^{-1} X^T\varSigma) (\theta^* + (X^T X)^{-1} X^T\varSigma)^T ]\\
	&=E[\theta^* \theta^{*T} + (X^T X)^{-1} X^T \varSigma \theta^{*T} + \theta^* \varSigma^T X (X^T X)^{-1} + (X^T X)^{-1} X^T \varSigma \varSigma^T X (X^T X)^{-1}]
	\end{split}
	\end{equation*}
	$\varSigma$ is the covariance matrix generated by $\epsilon$ and each $\epsilon^{(i)}$is an independent random variable drawn from a normal distribution with zero mean and variance $\sigma^2$. Therefore the expectation of $\varSigma$ is zero. $\varSigma$ is also independent to $X$ and $\theta^*$, $\varSigma = \sigma^2 I$, where $I$ is the identity matrix. Therefore
	$$ E[\theta^2]= (\theta^*)^2 + \sigma^2 I (X^T X)^{-1}$$
	Then we have 
	\begin{equation*}
	\begin{split}
	Var(\theta) &= E[\theta^2] -(E[\theta])^2\\
	&=(\theta^*)^2 + \sigma^2 I (X^T X)^{-1} -(\theta^*)^2\\
	& =  (X^T X)^{-1} \sigma^2
	\end{split}
	\end{equation*}
\end{proof}
\end{enumerate}



\end{document}


