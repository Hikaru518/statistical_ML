\documentclass[]{book}

%These tell TeX which packages to use.
\usepackage{array,epsfig}
\usepackage{amsmath}
\usepackage{amsfonts}
\usepackage{amssymb}
\usepackage{amsxtra}
\usepackage{amsthm}
\usepackage{mathrsfs}
\usepackage{color}
\usepackage{graphicx}
\usepackage{float}
\usepackage{booktabs}

%Algorithm packages
\usepackage{algorithm}  
\usepackage{algpseudocode}  
\usepackage{amsmath}  
\renewcommand{\algorithmicrequire}{\textbf{Input:}}  % Use Input in the format of Algorithm  
\renewcommand{\algorithmicensure}{\textbf{Output:}} % Use Output in the format of Algorithm  
%Here I define some theorem styles and shortcut commands for symbols I use often
\theoremstyle{definition}
\newtheorem{defn}{Definition}
\newtheorem{thm}{Theorem}
\newtheorem{cor}{Corollary}
\newtheorem*{rmk}{Remark}
\newtheorem{lem}{Lemma}
\newtheorem*{joke}{Joke}
\newtheorem{ex}{Example}
\newtheorem*{soln}{Solution}
\newtheorem{prop}{Proposition}

\newcommand{\lra}{\longrightarrow}
\newcommand{\ra}{\rightarrow}
\newcommand{\surj}{\twoheadrightarrow}
\newcommand{\graph}{\mathrm{graph}}
\newcommand{\bb}[1]{\mathbb{#1}}
\newcommand{\Z}{\bb{Z}}
\newcommand{\Q}{\bb{Q}}
\newcommand{\R}{\bb{R}}
\newcommand{\C}{\bb{C}}
\newcommand{\N}{\bb{N}}
\newcommand{\M}{\mathbf{M}}
\newcommand{\m}{\mathbf{m}}
\newcommand{\MM}{\mathscr{M}}
\newcommand{\HH}{\mathscr{H}}
\newcommand{\Om}{\Omega}
\newcommand{\Ho}{\in\HH(\Om)}
\newcommand{\bd}{\partial}
\newcommand{\del}{\partial}
\newcommand{\bardel}{\overline\partial}
\newcommand{\textdf}[1]{\textbf{\textsf{#1}}\index{#1}}
\newcommand{\img}{\mathrm{img}}
\newcommand{\ip}[2]{\left\langle{#1},{#2}\right\rangle}
\newcommand{\inter}[1]{\mathrm{int}{#1}}
\newcommand{\exter}[1]{\mathrm{ext}{#1}}
\newcommand{\cl}[1]{\mathrm{cl}{#1}}
\newcommand{\ds}{\displaystyle}
\newcommand{\vol}{\mathrm{vol}}
\newcommand{\cnt}{\mathrm{ct}}
\newcommand{\osc}{\mathrm{osc}}
\newcommand{\LL}{\mathbf{L}}
\newcommand{\UU}{\mathbf{U}}
\newcommand{\support}{\mathrm{support}}
\newcommand{\AND}{\;\wedge\;}
\newcommand{\OR}{\;\vee\;}
\newcommand{\Oset}{\varnothing}
\newcommand{\st}{\ni}
\newcommand{\wh}{\widehat}

%Pagination stuff.
\setlength{\topmargin}{-.3 in}
\setlength{\oddsidemargin}{0in}
\setlength{\evensidemargin}{0in}
\setlength{\textheight}{9.in}
\setlength{\textwidth}{6.5in}
\pagestyle{empty}



\begin{document}


\begin{center}
{\Large COMP 540 \hspace{0.5cm} HW 6}\\
\textbf{Peiguang Wang, Xinran Zhou}\\ %You should put your name here
Due: 4/20/2018 %You should write the date here.
\end{center}

\vspace{0.2 cm}


\subsection*{1: Hidden Markov Models}
\begin{enumerate}
	\item Identifying the set of observables $O$, the set of hidden states $S$ and th parameters $\lambda = [\pi, a, b]$, the initial state distribution, the transition matrix and the emission matrix. Draw the corresponding graphical model.
	\begin{soln}
		The set of observables is $O = \{low, medium, high\}$. The set of hidden states is $S = \{healthy, unhealthy\}$
		
		The parameters $\lambda$, here we use H and N to represent hidden states, and L, M and H to represent observables low, medium and high.
		
		The transition matrix $a$:
		\begin{tabular}{ccc}
			\toprule
			State&H&N\\
			\midrule
			H&0.8&0.2\\
			N&0.2&0.8\\
			\bottomrule
		\end{tabular}
	
		The emission matrix $b$: 
		\begin{tabular}{cccc}
			\toprule
			O/S&L&M&H\\
			\midrule
			H&0.5&0.3&0.2\\
			N&0.3&0.3&0.4\\
			\bottomrule
		\end{tabular} 
		
		The initial state distribution $\pi = [0.5,0.5]$
		
		
	\end{soln}
	\item Calculate the probability that the patient is healthy at $ t=2$ given that the test readings at $t = 1$ and $t = 2 $ are both low.
	\begin{soln}
		What we want to find is $p(x_2|L,L)$, so we can use filtering to calculate that. When $t = 1$
		$$\alpha_1(H) = b_H(e_1) \sum_{i =0}^{n}\alpha_0(i) a_{iH} = 0.5\times0.5 = 0.25$$
		$$\alpha_1(N) = 0.3 \times 0.5 = 0.15$$
		also we can calculate when $t = 2$
		$$\alpha_2(H) = 0.5 \times 0.23 = 0.115$$
		$$\alpha_2(N) = 0.3 \times 0.17 = 0.051$$
		so the probability that the patient is healthy at $ t=2$ given that the test readings at $t = 1$ and $t = 2 $ is that 
		$$p(x_2 = H|L,L) = \frac{0.115}{0.115+0.051} = 0.693$$
		
	\end{soln}
	\item What is the most likely state sequence for $t = 0,1,2$ given the evidence from the previous subpart?
	\begin{soln}
		H-H-H
	\end{soln}
\end{enumerate}
	

\end{document}


